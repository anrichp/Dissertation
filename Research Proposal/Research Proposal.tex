\documentclass[12pt]{article}
\usepackage{times}
\usepackage[style=authoryear, backend=biber]{biblatex}
\linespread{1.5}

\title{Research Proposal}
\author{Anrich Potgieter}
\date{21-04-2022}

\begin{document}

\maketitle
\tableofcontents

\section{Research Summary}
A blockchain representing banking transactions for auditing and regulatory purposes.
This would obviously only cover one element of the audit process however bank and cash would benefit from such a system.
There are specific line items that don't require human judgetment such as **What?** The transaction for that line item is stored on the blockchain and the supporting evidence for the transaction is added to the transaction which then represents truth once verified.

The problem statement is that there is a need for something to validate every transaction to ensure that the financial statements are true. To help with validation process, you use the Blockchain that is a leger of all transactions such as operating expenses, for any no. Judgemental line items. Each transaction in that line item. The invoice and proof of payment is attached, the node verifies the snark contract and validates the transaction data.

Typically the auditor will need to receive data dumps from the SAP system however in this case the auditor would have access to the blockchain as a means of accessing the data that needs to be included in the audit.
\section{Methodology}

\section{Key Literature}

\section{Human Participants}

\section{Timeline}

\end{document}