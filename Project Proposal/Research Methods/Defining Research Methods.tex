\documentclass[12pt]{article}
\usepackage{times}
\usepackage[style=authoryear, backend=biber]{biblatex}
\addbibresource{ResearchMethods.bib}
\linespread{1.5}

\title{Defining Research Methods}
\author{Anrich Potgieter}
\date{04/04/2022}

\begin{document}

\maketitle
\tableofcontents

\section{Methods and Methodology}

Action Research and Experimental Research.

\subsection{Overview of Approach}

For my dissertation project, I anticipate that I will use a combination of both action research and experimental research methods.
\\\\
\autocite{dawsonChapterResearchMethods2015} state that action research involves defining a specific problem, then working towards a solution along with a subject or organisation.
Furthermore, \autocite{blairChapterChoosingMethodology2016} states that action research is considered a methodology or an approach to "find solutions to practical problems." 
Although action research falls within qualitative research, I suspect that it may allow for flexibility when developing software as the approach in exploration to solve a specific problem. In the case of my proposed project, I am attempting to discover whether blockchains can interoperate with centralised systems.
\\\\
In addition to action research, the \autocite{dawsonChapterResearchMethods2015} highlights that development projects typically fall under the experiment method, where the researcher investigates the "causal relationships using tests controlled by yourself."
From my initial understanding, I suppose that I would develop a series of tests to establish whether the software I have developed meets the requirements of the proposed system design.

\printbibliography

\end{document}