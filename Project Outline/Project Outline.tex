\documentclass[12pt]{article}
\usepackage{times}
\usepackage[style=authoryear, backend=biber]{biblatex}
\addbibresource{ProjectOutline.bib}
\linespread{1.5}

\title{Project Outline}
\author{Anrich Potgieter}
\date{12-03-2022}

\begin{document}

\maketitle
\tableofcontents

\section{Research area and working title}
\subsection{Research Area}
Blockchain

\subsection{Working Title}
Connecting Blockchain of Blockchains to centralised systems and decentralised ledger systems (also known as blockchain technology)


\section{Abstract}
\autocite{belchiorSurveyBlockchainInteroperability2021}
state that "blockchains of blockchains are frameworks that provide reusable data, network consensus, incentive,
and contract layers for creating application-specific blockchains (customised blockchains) that interoperate between each other."
They further outline the current gap in the research regarding connecting existing centralised systems such as traditional banking
systems and decentralised ledger systems such as blockchains to other blockchains.
\\\\
Interoperability between blockchains has been an emerging field of
research for some time. It has gained significant traction since the whitepaper by Gavin Wood was released in 2016 that introduced Polkadot,
a multi-chain development environment allowing for interoperability between blockchains using a mechanism known as a parachain or bridge \autocite{woodPOLKADOTVISIONHETEROGENEOUS2016}.
What currently requires further exploration is how to connect centralised systems such as VisaNet Visa's electronic payments network to existing blockchain technologies \autocite{VisaNetTechnologyVisa}.
Connecting centralised banking systems to blockchains could be possible using parachains along with substrate to leverage existing API's to create deep integrations into existing blockchain ecosystems \autocite{polkadotPolkadotDecoded20202021}.
\\\\
Visa in a recent research paper has outlined developments in their own attempt to achieve cross-chain interoperability using a universal payment channel that provides off chain payment mechanisms
supported by a blockchain ledger using smart contracts, hashlocks and timelocks \autocite{christodorescuUniversalPaymentChannels2021}. Some criticality is required regarding their solution to determine whether their off chain solution follows the paradigms
outlined in the Satoshi Nakamoto white paper where trust of a third party is not required and the user is in control of their assets \autocite{nakamotoBitcoinPeertoPeerElectronic}.
\\\\
In my research I want to evaluate the existing interoperability solutions that are discussed in the available literature and establish the feasability of connecting blockchains to existing centralised systems.
Furthermore I want to explore interoperability frameworks such as Polkadot along with Substrate to determine whether tools avialable are sufficient to achieve Interoperability with centralised systems and decentralised ledger systems.

\section{Proposed Research Questions}
\paragraph{RQ1} What is the state of existing cross-chain interoperabiltity solutions and do they trully achieve the invisioned outcome of an internet of blockchains?
\paragraph{RQ2} Do existing cross-chain frameworks such as Polkadot coupled with substrate provide a means to connect centralised systems to blockchains?
\paragraph{RQ3} Would the proposed Visa Universal Payment Channels support interoperabiltity with existing blockchain of blockchain networks?

\section{Proposed Aims and Objectives}
\begin{itemize}
    \item Provide an overview of cross-chain interoperability.
    \item Determine whether parachains and substrate can connect to centralised systems.
    \item Propose a solution to connecting centralised systems to blockchains using interoperability frameworks.
    \item Provide an overview of existing Polkadot bridges and their effectiveness in connecting blokchains to blockchains.
\end{itemize}

\section{Proposed Research Design}
\paragraph{Action research}\mbox{} \\
For this dissertation I want to use an action research methodology \autocite{10.5555/2842927}. I find that the structure of action research would provide a
less structures more experiemental approach to my research. Action research typically allows the researcher to actively attempt to solve a problem and document their findings thereafter.

\section{Artefact/s that can be created}
\begin{itemize}
    \item Diagrams providing an overview of all known interoperability framweworks promoting cross-chain interoperability.
    \item Development of a Polkadot bridge providing interoperability between centralised systems and blockchain technologies.
    \item Development of a Polkadot parachain connecting a sharded chain to a web 2.0 technology.
\end{itemize}

\printbibliography

\end{document}