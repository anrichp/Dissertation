\documentclass[12pt]{article}
\usepackage{times}
\usepackage[style=authoryear, backend=biber]{biblatex}
\addbibresource{ProjectOutline.bib}
\linespread{1.5}

\title{Project Outline}
\author{Anrich Potgieter}
\date{12-03-2022}

\begin{document}

\maketitle
\tableofcontents

\section{Research area and working title}
\subsection{Research Area}
\subsubsection{Blockchain}

\subsection{Working Title}
\subsubsection{Connecting Blockchain of Blockchains to centralised systems and decentralised ledger systems (also known as blockchain technology)}


\section{Abstract}
\autocite{belchiorSurveyBlockchainInteroperability2021}
state that "blockchains of blockchains are frameworks that provide reusable data, network consensus, incentive, 
and contract layers for creating application-specific blockchains (customised blockchains) that interoperate between each other."
They further outline the current gap in the research regarding connecting existing centralised systems such as traditional banking
systems and decentralised ledger systems such as blockchains to other blockchains. Interoperability between blockchains has been an emerging field of
research for some time. It has gained significant traction since the whitepaper by Gavin Wood was released in 2016 that introduced Polkadot, 
a multi-chain development environment allowing for Interoperability between blockchains using a mechanism known as a parachain or bridge \autocite{woodPOLKADOTVISIONHETEROGENEOUS2016}.
What currently requires further exploration is how to connect centralised systems such as VisaNet Visa's electronic payments network to existing blockchain technologies \autocite{VisaNetTechnologyVisa}.
Connecting centralised banking systems to blockchains could be possible using parachains along with substrate to leverage existing API's to create deep integrations into existing blockchain ecosystems \autocite{polkadotPolkadotDecoded20202021}.
\\\\ 
Visa in a recent research paper has outlined recent developments in their own attempt to achieve cross-chain Interoperability using a universal payment channel that provides off chain payment mechanisms 
supported by a blockchain ledger using haslocks and timelocks \autocite{christodorescuUniversalPaymentChannels2021}. Some criticality is required regarding their solution to determine whether their off chain solution follows the paradigms
outlined in the Staoshi Nakamoto white paper where trust of a third party is not required and the user is in control of their assets \autocite{nakamotoBitcoinPeertoPeerElectronic}.
\section{Proposed Research Problem and Research Question}
\subsubsection{Research Problem}

\subsubsection{Research Questions}
\paragraph{RQ1} Do the current cross-chain solutions provide mechanisms to connect to centralised and decentralised ledger systems?
\paragraph{RQ2} Does the proposed Visa universal payment channels solution support bridging solutions?
\paragraph{RQ3} Are off-chain solutions problematic for the paradigm under which blockchains operate?

\section{Proposed Aims and Objectives}

\section{Proposed Research Design}
\paragraph{Action research}\mbox{} \\
I want to use action research.

\section{Artefact/s that can be created}

\printbibliography

\end{document}