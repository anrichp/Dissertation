\documentclass[12pt]{article}
\usepackage{times}
\usepackage[style=authoryear, backend=biber]{biblatex}
\addbibresource{ProjectOutline.bib}
\linespread{1.5}

\title{Project Outline}
\author{Anrich Potgieter}
\date{12-03-2022}

\begin{document}

\maketitle
\tableofcontents

\section{Research area and working title}
\subsection{Research Area}
\subsubsection{Blockchain}

\subsection{Working Title}
\subsubsection{Connecting Blockchain of Blockchains to centralised systems and decentralised ledger systems (also known as blockchain technology)}


\section{Abstract}
\autocite{belchiorSurveyBlockchainInteroperability2021}
state that "blockchains of blockchains are frameworks that provide reusable data, network consensus, incentive, 
and contract layers for creating application-specific blockchains (customised blockchains) that interoperate between each other."
They further outline that there is a current gap in the research regarding connecting existing centralised systems such as traditional banking
systems as well as decentralised ledger systems such as blockchains to other blockchains.

\section{Proposed Research Problem and Research Question}
\subsubsection{Research Problem}

\subsubsection{Research Questions}
\paragraph{RQ1} Do the current cross-chain solutions provide mechanisms to connect to centralised and decentralised ledger systems?
\paragraph{RQ2} Does the proposed Visa universal payment channels solution support bridging solutions?
\paragraph{RQ3} Are off-chain solutions problematic for the paradigm under which blockchains operate?

\section{Proposed Aims and Objectives}

\section{Proposed Research Design}
\paragraph{Action research}\mbox{} \\
I want to use action research.

\section{Artefact/s that can be created}

\printbibliography

\end{document}