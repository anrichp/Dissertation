\section{Defining Blockchain Technology}

\subsection{Background}

Blockchain technology reaches back far further than the inception of Bitcoin, and we can see some of the first implementations appearing in 1998.
In a 1998 white paper titled bmoney, we see some of the earliest building blocks of cryptocurrencies and the adoption of blockchain technologies \autocite{daiBmoney1998}.
Wei Dai outlines some cornerstone concepts that would later inspire Satoshi Nakamoto to create Bitcoin. Wei begins to outline a form of Zero Knowledge proof where two parties involved in an exchange or transaction use pseudonyms in the form of public keys to identify themselves within the context of a transaction \autocite{ZeroknowledgeProofsEthereum2022}. Furthermore, Wei begins laying the foundation of cryptographically complex puzzles that are solved to determine the value of the currency transferred. The concepts mentioned above would eventually lead to one of the crucial components of blockchains known as proof of work.\par
Further to the cryptographic puzzles introduced by Wei Dai in 2002, we see the emergence of a white paper by Adam Back titled hashcash \autocite{backHashcashDenialService2002}. Back, originally envisioned these concepts to solve denial of service attacks on email servers where communication over these email protocols would come at a greater cost of computational power. Later Back realised that this denial of service concept would effectively translate into a proof-of-work function where this function would create a token representing the computational complexity required to solve the hash.\par
As seen above, proof-of-work is an essential cornerstone of blockchain technology and is historically significant to one of the most important blockchains in history, bitcoin. 2004 was potentially the most crucial year in the history of blockchain technology; in a white paper titled RPOW - Reusable Proofs of Work by Hal Finny, we observe a piece of client software that creates an RPOW token cryptographically signed by the client's private key \autocite{finneyRPOWReusableProofs}. The token mentioned before is stored on a secure server that identifies the stored token ownership by the private key. If the private key owner wishes to transfer this token to another user, they sign a transfer order which is stored as a public key; the server then assigns the transferred token to the new owner's private key.\par
In 2008 we observed a culmination of various concepts seen within the cryptography development space with the emergence of the infamous blockchain, Bitcoin. In the white paper by Satoshi Nakamoto, the fundamental components of blockchain technology appear in a concise collection of computing concepts that facilitate a blockchain where a user can store electronic cash without a third-party financial institution \autocite{nakamotoBitcoinPeertoPeerElectronic2008}. The Bitcoin whitepaper introduces the fundamental components of a blockchain with an overview of how these components work together. The fundamental components or concepts outlined are transactions, timestamp server, proof-of-work, networks (Nodes), incentives and payment verification. The concepts mentioned above provided a foundation for future blockchain development and research directions.\par
In 2014 sometime after the initial release of Bitcoin, a white paper by Vitalik Buterin surfaced titled Ethereum, where Buterin outlined a vision for the future of blockchain development \autocite{vitalikEthereumWhitepaper2014}. Buterin proposed the expansion of the fundamental components of Bitcoin to create a development environment facilitating the creation of "consensus-based applications". Furthermore, Buterin surmises a new addition to the standard blockchain: the smart contract. The aforementioned smart contract is defined as a "computerised transaction protocol" that defines a series of contractual conditions once met, a transaction is complete \autocite{yagaBlockchainTechnologyOverview2018}. The invention of smart contacts has significantly repositioned blockchain technology to solve various complex trust-based scenarios in many industries.\par
The concepts outlined before are essential concepts required to create a blockchain. \autocite{dipierroWhatBlockchain2017} states that blockchains solve the problem of establishing trust in a "distributed system". Furthermore, \autocite{yagaBlockchainTechnologyOverview2018} outlines that blockchains provide reliable trust mechanisms using a "tamper resistant digital ledger." The ledger relies on a peer-to-peer network with a series of nodes which synchronously replicate data across the network to ensure data availability across the network \autocite{butijnBlockchainsSystematicMultivocal2020}. Hashing network time stamps into a continuous chain of "hash-based proof-of-work" blocks ensure the reliability of the data, furthermore it would be impossible to make changes to data without asking nodes in the network to re-do the proof of work.\par In the sections below, we will explore the components and concepts of blockchain technology to reveal future development opportunities and use cases for blockchains in banking organisations.

\subsection{Types of Blockchains}

Blockchains typically fall into three succinct categories, permissionless, permissioned and consortium. Historically permissionless blockchains are most prevalent; however, permissioned blockchains have come to the forefront as the need for blockchain technology has emerged in various industries with different use cases \autocite{yagaBlockchainTechnologyOverview2018}. A further extension of private or permissioned blockchains is consortium blockchains where authenticated participants are decided by a consortium or organisation \autocite{leibleReviewBlockchainTechnology2019}.

\subsubsection{Permissionless}

Permissionless blockchains are those in which any individual can participate in the network by reading and writing to it without authorisation \autocite{yagaBlockchainTechnologyOverview2018}.
Bitcoin is the first example of a permissionless blockchain; however, when Nakamoto created Bitcoin, standardised terms for blockchain types were not yet in use. In the Bitcoin whitepaper, Nakamoto addresses the privacy of traditional banking systems where a trusted third party verifies transactions using information about each party to complete the transaction \autocite{nakamotoBitcoinPeertoPeerElectronic2008}. Nakamoto proposes an alternative where the identity of a network participant identifies itself using a public key rather than identifiable information, as seen in traditional banking systems. 

\subsubsection{Permissioned}

Permissioned blockchains also known as private blockchains rely on a third party to select participants or nodes in the network \autocite{imteajIntroductionBlockchainTechnology2021}.\autocite{yagaBlockchainTechnologyOverview2018} states that permissioned blockchains provide "finer-grained" permission controls for blockchain use cases where organisations require access control, limiting network participation by authenticating a node. Furthermore, \autocite{leibleReviewBlockchainTechnology2019} outlines that permissioned blockchains extend further than a means of authentication on the network. Leible states that the categorisation of blockchain types also extends to the governance and consensus mechanisms; additionally, in permissioned blockchains, node participation is not equal. All nodes in the network are often a collection of nodes within an organisation which is in contrast to permissionless blockchains such as Bitcoin, where participation is unlimited and anonymous. Leible further outlines the categorisation of user types or participants in a blockchain network; users are defined as "user" and "committee user". The users or nodes in the network as the name of the blockchain implies can perform certain restricted actions on the blockchain \autocite{butijnBlockchainsSystematicMultivocal2020} \autocite{rajasekaranComprehensiveSurveyBlockchain2022}.

\subsubsection{Consortium}

Consortium blockchains are an extension of permissioned blockchains where nodes and participants are required to authenticate on the network, although consortiums allow nodes from external organisations to participate in the network \autocite{butijnBlockchainsSystematicMultivocal2020}. Furthermore, a consortium blockchain facilitates distributed validation where the consortium identifies nodes to verify transactions. This is particularly useful for organisations to agree upon effective data-sharing governance upheld by the blockchain \autocite{imteajIntroductionBlockchainTechnology2021}. Consortiums are further categorised by node participation if the authenticated nodes on the network do not change, this is known as a "static consortium", whereas if nodes change over time the consortium is known as an "agile consortium" \autocite{ruotiBlockchainTechnologyWhat2019}. 

\subsection{Blockchain Components}

In this section, we will investigate the critical components that make up a blockchain. Blockchain technology leverages various computer science disciplines however, a majority of the foundational concepts rely on cryptography. 

\subsubsection{Cryptographic Hash Functions}

Cryptography forms the foundation of blockchain technology and facilitates a means to define the rules of the blockchain so that transactions are tamper-proof and auditable \autocite{narayananBitcoinCryptocurrencyTechnologies2016} \autocite{imteajIntroductionBlockchainTechnology2021} \autocite{leibleReviewBlockchainTechnology2019}. Furthermore, cryptographic hash functions are extensively used in cryptocurrency-based blockchains as a means to generate puzzles which miners solve to receive currency incentives \autocite{yagaBlockchainTechnologyOverview2018}. \autocite{narayananBitcoinCryptocurrencyTechnologies2016} defines a hash function as a mathematical function with a series of properties:

\begin{itemize}
    \item an input can be a string of any length
    \item typically outputs a fixed length of 256 bits 
    \item the output of a hashed string is "efficiently computable"
\end{itemize}

In the context of a blockchain, a hash function must be cryptographically secure, i.e a cryptographic hash function. A cryptographic hash function has three properties "collision resistance, hiding and puzzle friendliness" \autocite{narayananBitcoinCryptocurrencyTechnologies2016}. Below is a further exploration of the aforementioned properties.

\paragraph{Property 1: Collision Resistance}

A cryptographic hash function must be collision resistant to ensure the function is secure \autocite{dipierroWhatBlockchain2017}. A cryptographic collision is one where two unique inputs produce the same output \autocite{narayananBitcoinCryptocurrencyTechnologies2016} \autocite{yagaBlockchainTechnologyOverview2018}. For example, an insecure hash function may generate an identical hash where the same characters are in a different order \autocite{nakovPracticalCryptographyDevelopers2018}. To prevent a collision we require a collision detection algorithm, however, the probability of discovering a collision is astronomically small when using a cryptographic hash function such as SHA-256, in this case, due to computational cost, we place our trust in the complexity of the hash function and assume that a collision is unlikely \autocite{yagaBlockchainTechnologyOverview2018}.

\paragraph{Property 2: Hiding}

The hiding property of a cryptographic hash function supposes that it is impossible to calculate the input of the hash function given the hash output \autocite{narayananBitcoinCryptocurrencyTechnologies2016}. In a blockchain, we make use of a commitment scheme to achieve the hiding property. A commitment uses two algorithmic methods, one called the commitment and verification, the commitment takes an input message and a secret unique random number called a nonce (which is used once) and outputs a commitment hash \autocite{yagaBlockchainTechnologyOverview2018}. The commitment hash is verified by the verification method where the method receives the commitment, message and nonce and will return true if "com = = commit(msg, nonce)" \autocite{narayananBitcoinCryptocurrencyTechnologies2016}. The hiding property is fulfilled when given the commitment it is improbable to find the message.

\paragraph{Property 3: Puzzle Friendliness}

\autocite{narayananBitcoinCryptocurrencyTechnologies2016} states A puzzle-friendly hash function is difficult to solve using a strategy, rather the puzzle is solved through brute force. To define a puzzle we use the expression "\(H(id \ || \ x) \in Y \)":

\begin{itemize}
    \item H is the hash function
    \item id is the id of the puzzle
    \item Y is a target set, i.e the difficulty of the puzzle area that is searched.
\end{itemize}

\subsubsection{Hash Pointers and Data Structures}

\subsubsection{Digital Signatures}

Digital signatures are the primary cryptographic mechanism to prove ownership of a blockchain token \autocite{nakamotoBitcoinPeertoPeerElectronic2008}. 

\subsubsection{Transactions}

Blockchain transactions are a concept popularised by Bitcoin, an owner of a Bitcoin or a fraction thereof transfers the currency to another participant by signing a hash of a previous transaction and the public key of the intended receiver to the end of the coin \autocite{nakamotoBitcoinPeertoPeerElectronic2008} \autocite{assiaColoredCoinsWhite2015}. Within the Ethereum ecosystem transactions are referred to as messages, these messages are fundamentally similar to Bitcoin transactions however, Ethereum transactions facilitate smart contracts. Furthermore, Ethereum smart contacts return a response once a contract is completed which allows for complex functions \autocite{vitalikEthereumWhitepaper2014}. The concept that emerges from understanding transactions is what defines a blockchain, a chain of blocks. Bitcoin was the first to solve the double-spending problem, using a historical ledger of all transactions that are tamper-free. The use of a historical ledger provides a means of verifying whether a coin was previously spent and thus preventing a double-spending problem \autocite{narayananBitcoinCryptocurrencyTechnologies2016}. 

\subsubsection{Asymmetric-Key Cryptography}

\subsubsection{Addresses}

\subsubsection{Ledgers}

\subsubsection{Blocks}

\subsubsection{Chaining Blocks}

\subsection{Consensus}

\subsubsection{Proof of Work (PoW)}

\subsubsection{Proof of Stake (PoS)}

\subsubsection{Delegated Proof of Stake (DPoS)}

\subsubsection{Proof of Elapsed Time (PoET)}

\subsubsection{Practical Byzantine Fault Tolerance (PBFT)}

\subsection{Smart Contracts}

\section{Organisational Interoperability}

\section{Facilitating Interoperability using Blockchain Technology}

\section{Blockchain Technology in Banking Organisations}

\subsection{Permissioned Blockchain Networks}

\section{Blockchain Data Storage and Retrieval}