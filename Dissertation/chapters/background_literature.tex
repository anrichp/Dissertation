\section{Defining Blockchain Technology}

\subsection{Background}

Blockchain technology reaches back far further than the inception of Bitcoin, and we can see some of the first implementations appearing in 1998.
In a 1998 white paper titled bmoney, we see some of the earliest building blocks of cryptocurrencies and the adoption of blockchain technologies \autocite{daiBmoney1998}.
Wei Dai outlines some cornerstone concepts that would later inspire Satoshi Nakamoto to create Bitcoin. Wei begins to outline a form of Zero Knowledge proof where two parties involved in an exchange or transaction use pseudonyms in the form of public keys to identify themselves within the context of a transaction \autocite{ZeroknowledgeProofsEthereum2022}. Furthermore, Wei begins laying the foundation of cryptographically complex puzzles that are solved to determine the value of the currency transferred. The concepts mentioned above would eventually lead to one of the crucial components of blockchains known as proof of work.\par
Further to the cryptographic puzzles introduced by Wei Dai in 2002, we see the emergence of a white paper by Adam Back titled hashcash \autocite{backHashcashDenialService2002}. Back, originally envisioned these concepts to solve denial of service attacks on email servers where communication over these email protocols would come at a greater cost of computational power. Later Back realised that this denial of service concept would effectively translate into a proof-of-work function where this function would create a token representing the computational complexity required to solve the hash.\par
As seen above, proof-of-work is an essential cornerstone of blockchain technology and is historically significant to one of the most important blockchains in history, bitcoin. 2004 was potentially the most crucial year in the history of blockchain technology; in a white paper titled RPOW - Reusable Proofs of Work by Hal Finny, we observe a piece of client software that creates an RPOW token cryptographically signed by the client's private key \autocite{finneyRPOWReusableProofs}. The token mentioned before is stored on a secure server that identifies the stored token ownership by the private key. If the private key owner wishes to transfer this token to another user, they sign a transfer order which is stored as a public key; the server then assigns the transferred token to the new owner's private key.\par
In 2008 we observed a culmination of various concepts seen within the cryptography development space with the emergence of the infamous blockchain, Bitcoin. In the white paper by Satoshi Nakamoto, the fundamental components of blockchain technology appear in a concise collection of computing concepts that facilitate a blockchain where a user can store electronic cash without a third-party financial institution \autocite{nakamotoBitcoinPeertoPeerElectronic}. The Bitcoin whitepaper introduces the fundamental components of a blockchain with an overview of how these components work together. 

\subsection{Types of Blockchains}

\subsubsection{Permissionless}

\subsubsection{Permissioned}

\subsubsection{Consortium}

\subsection{Blockchain Components}

\subsubsection{Cryptographic Hash Functions}

\subsubsection{Transactions}

\subsubsection{Asymmetric-Key Cryptography}

\subsubsection{Addresses}

\subsubsection{Ledgers}

\subsubsection{Blocks}

\subsubsection{Chaining Blocks}

\subsection{Consensus}

\subsubsection{Proof of Work (PoW)}

\subsubsection{Proof of Stake (PoS)}

\subsubsection{Delegated Proof of Stake (DPoS)}

\subsubsection{Proof of Elapsed Time (PoET)}

\subsubsection{Practical Byzantine Fault Tolerance (PBFT)}

\subsection{Smart Contracts}

\section{Organisational Interoperability}

\section{Facilitating Interoperability using Blockchain Technology}

\section{Blockchain Technology in Banking Organisations}

\subsection{Permissioned Blockchain Networks}

\section{Blockchain Data Storage and Retrieval}