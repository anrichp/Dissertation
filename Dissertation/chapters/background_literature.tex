\section{Defining Blockchain Technology}

\subsection{Background}

Blockchain technology reaches back far further than the inception of Bitcoin, and we can see some of the first implementations stretching back as far as 1998.
In a 1998 white paper titled bmoney, we see some of the earliest building blocks of cryptocurrencies and the adoption of blockchain technologies \autocite{daiBmoney1998}.
Wei Dai outlines some cornerstone concepts that would later inspire Satoshi Nakamoto to create Bitcoin. Wei begins to outline a form of Zero Knowledge proof where two parties involved in an exchange or transaction use pseudonyms in the form of public keys to identify themselves within the context of a transaction \autocite{ZeroknowledgeProofsEthereum2022}.

\subsection{Types of Blockchains}

\subsubsection{Permissionless}

\subsubsection{Permissioned}

\subsubsection{Consortium}

\subsection{Blockchain Components}

\subsubsection{Cryptographic Hash Functions}

\subsubsection{Transactions}

\subsubsection{Asymmetric-Key Cryptography}

\subsubsection{Addresses}

\subsubsection{Ledgers}

\subsubsection{Blocks}

\subsubsection{Chaining Blocks}

\subsection{Consensus}

\subsubsection{Proof of Work (PoW)}

\subsubsection{Proof of Stake (PoS)}

\subsubsection{Delegated Proof of Stake (DPoS)}

\subsubsection{Proof of Elapsed Time (PoET)}

\subsubsection{Practical Byzantine Fault Tolerance (PBFT)}

\subsection{Smart Contracts}

\section{Organisational Interoperability}

\section{Facilitating Interoperability using Blockchain Technology}

\section{Blockchain Technology in Banking Organisations}

\subsection{Permissioned Blockchain Networks}

\section{Blockchain Data Storage and Retrieval}